%+++++++++++++++++++++++++++++++++++++++++++
\documentclass[conference]{IEEEtran}
%+++++++++++++++++++++++++++++++++++++++++++
% Added to commands
\input epsf
\usepackage{graphicx}

%+++++++++++++++++++++++++++++++++++++++++++
% correct bad hyphenation here
\hyphenation{op-tical net-works semi-conduc-tor IEEEtran}
\begin{document}

%+++++++++++++++++++++++++++++++++++++++++++
% paper title
\title{\LARGE Network Analysis of 'MiBici' Public Bike Service in Guadalajara's Metropolitan Area}

\author{Henry Marie MONT and Matteo MATONE \\
University of Tartu, Department of Computer Science \\
Tartu, Estonia \\
\texttt{henry.marie.mont@ut.ee, matone@ut.ee}}

\maketitle

\begin{abstract}

This paper presents a comprehensive network analysis of the public shared bike service known as "MiBici" in Guadalajara's metropolitan area (ZMG) in Jalisco, Mexico, spanning from December 2014 to January 2024. We explore the network structure, dynamics, and user behaviors of the "MiBici" system using different network analysis techniques. Through data preprocessing, network construction, and analysis scripts, we uncover insights into the spatial distribution of bike stations, traffic flows, user mobility patterns, and system efficiency. Our findings contribute to enhancing the understanding of public bike-sharing systems and provide valuable insights for urban planning and transportation management.

\end{abstract}
\IEEEoverridecommandlockouts
\begin{keywords}
Public bike-sharing, Network analysis, Complex network theory, Urban transportation, Mexico, Guadalajara.
\end{keywords}

\section{Introduction}

Public bike-sharing systems have become increasingly popular in urban areas as a sustainable and convenient mode of transportation. These systems provide users with access to bicycles for short trips, contributing to reducing traffic congestion and promoting healthier lifestyles. Understanding the dynamics of public bike-sharing systems, including network structure, user behaviors, and system efficiency, is crucial for optimizing service operations and guiding urban planning efforts.

In this paper, we focus on the network analysis of the "MiBici" public bike service in Guadalajara's metropolitan area (ZMG) in Jalisco, Mexico. By exploring the spatial distribution of bike stations, traffic flows, and user mobility patterns, we aim to gain insights into the functioning of the "MiBici" system and its implications for urban transportation in Guadalajara.

\section{Related work}

The exploration of public bicycle systems through network analysis has garnered significant attention in recent literature, offering insights into system structure, dynamics, and user behaviors. Wei, Xu, and Ma \cite{wei2019exploring} demonstrated the applicability of complex network theory and shortest path analysis in characterizing the public bicycle network structure in Yixing, China. Yao et al. \cite{yao2019analysis} focused on analyzing urban bike-sharing systems using real-time data from the Nanjing public bicycle system. Dobrzyńska and Dobrzyński \cite{dobrzynska2017structure} conducted a comprehensive analysis of the BiKeR public bike-sharing system in Białystok, Poland, emphasizing the dynamics of network topology changes and station location choices. Jurdak \cite{jurdak2013impact} investigated the impact of cost and network topology on urban mobility using data from public bicycle share systems in two U.S. cities. Xiao et al. \cite{xiao2021demand} addressed the challenge of short-term demand prediction in public bike-sharing programs, proposing a spatio-temporal graph convolutional network (STGCN) approach. Builes-Jaramillo and Lotero \cite{builes-jaramillo2022spatial} conducted spatial-temporal network analysis of the public bicycle sharing system in Medellín, Colombia. 

These studies collectively contribute to advancing knowledge on public bicycle systems, offering insights into network structure, dynamics, user behaviors, and planning strategies, which are instrumental in promoting sustainable urban transportation.

\section{Dataset}

WORK IN PROGRESS

\section{Methodology}

WORK IN PROGRESS

\section{Results}

WORK IN PROGRESS

\section{Conclusion}

WORK IN PROGRESS

\begin{thebibliography}{1}

\bibitem{wei2019exploring}
Wei, Sheng, Jiangang Xu, and Haitao Ma. “Exploring Public Bicycle Network Structure Based on Complex Network Theory and Shortest Path Analysis: The Public Bicycle System in Yixing, China.” \textit{Transportation Planning and Technology} 42, no. 3 (2019): 293–307. doi:10.1080/03081060.2019.1576385.

\bibitem{yao2019analysis}
Yao, Yi, Yifang Zhang, Lixin Tian, Nianxing Zhou, Zhilin Li, and Minggang Wang. 2019. "Analysis of Network Structure of Urban Bike-Sharing System: A Case Study Based on Real-Time Data of a Public Bicycle System" \textit{Sustainability} 11, no. 19: 5425. https://doi.org/10.3390/su11195425.

\bibitem{dobrzynska2017structure}
Dobrzyńska, Ewa, and Dobrzyński, Maciej. "Structure and dynamics of a public bike-sharing system. Case study of the public transport system in Białystok" \textit{Engineering Management in Production and Services} 8, no.4 (2017): 59-66. https://doi.org/10.1515/emj-2016-0033.

\bibitem{jurdak2013impact}
Jurdak, Raja. "The Impact of Cost and Network Topology on Urban Mobility: A Study of Public Bicycle Usage in 2 U.S. Cities." \textit{PLoS ONE} 8, no. 11 (2013): e79396. https://doi.org/10.1371/journal.pone.0079396.

\bibitem{xiao2021demand}
Xiao, G., Wang, R., Zhang, C. et al. Demand prediction for a public bike sharing program based on spatio-temporal graph convolutional networks. \textit{Multimed Tools Appl} 80, 22907–22925 (2021). https://doi.org/10.1007/s11042-020-08803-y.

\bibitem{builes-jaramillo2022spatial}
Builes-Jaramillo, Alejandro, and Laura Lotero. "Spatial-temporal network analysis of the public bicycle sharing system in Medellín, Colombia." \textit{Journal of Transport Geography} 105 (2022): 103460. https://doi.org/10.1016/j.jtrangeo.2022.103460.

\end{thebibliography}

\section{Github}

WORK IN PROGRESS

\end{document}